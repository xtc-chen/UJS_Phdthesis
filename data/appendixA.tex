\appendixA{
\section{附录要求}
如果要修改,请去ujsthesis.cls文件,查找附录A和附录B这两个关键词,修改为需要的标题。

1.学位论文的附录部分
附录部分作为学位论文主体的补充项目,并不是必须的。主要包括:

(1) 正文内过分冗长的公式推导;

(2) 为他人阅读方便所需辅助性数学工具或重复性的图、表;

(3) 由于过分冗长而不宜在正文中出现的计算机程序清单。

附录的序号编排按附录A,附录B…编排,附录(例如附录B)内的顺序可按B.1,B.1.1,B.1.1.1的规律编排,其内图表按:图B1,图B2,表B1,表B2的规律编排。

附录可以单独装订,单独编排页码,封面上需注明论文正本的名称、作者等内容。如果附录与论文正本装在一起,则与正文连续编排页码。
\begin{equation}
	\left(\begin{array}{l}
		x \\
		y \\
	\end{array}\right)=f\left[\left\{\begin{array}{l}
		\xi \\
		\eta \\
	\end{array}\right\}\right]
\end{equation}

\begin{figure}[!ht]
	\centering
	{\includegraphics[scale=0.1]{Figures/logo/ujslogo.pdf}}
	\bicaption{江苏大学}{Jiangsu University}
	\label{Jiangsu University1}
\end{figure}

\begin{table}[h]
	\centering
	\renewcommand{\arraystretch}{1.5} %调整段落行距
	\setlength{\tabcolsep}{15pt} %调整宽度
	\caption{Parameters of flux weakening of the three motors HPM}
	\begin{tabular}{cccc}
		\toprule[1.5pt]
		参数 & 永磁同步电机 I & 永磁同步电机 II & 永磁同步电机 III \\ \midrule[0.75pt]
		初始转矩 \(T_{ini}\) & 27.7Nm & 12.8Nm & 8.6Nm\\
		初始转矩 \(T_{ini}\) & 27.7Nm & 12.8Nm & 8.6Nm \\ \bottomrule[1.5pt]  
	\end{tabular}
\end{table}


}
