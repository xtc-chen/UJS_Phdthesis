\appendixB{
如果要修改,请去ujsthesis.cls文件,查找附录A和附录B这两个关键词,修改为需要的标题。

1.学位论文的页码编排
学位论文依次按照中文题名页、英文题名页、论文独创性声明、论文版权使用授权书、中文摘要、英文摘要、目录、图表清单(必要时)、注释表(必要时)、引言(或绪论)、正文、结论、参考文献、致谢、在学期间发表的学术论文及其他科研成果、附录(必要时)的顺序编排。中文摘要页至注释表页用大写罗马数字连续编排页码;引言(或绪论)页至附录页用阿拉伯数字连续编排页码。页码位于页脚外侧。
学位论文的页眉从中文摘要开始,采用五号宋体字居中书写,奇数页写“江苏大学博(或硕)士学位论文”,偶数页写学位论文的题目。

2.学位论文的排版打印与装订
学位论文以A4(210×297mm)纸页面排版,双面打印。每一页的上方(天头)和左侧应分别留边25mm以上,下方(地脚)和右侧(切口)应分别留边20mm以上,用学校统一印制的学位论文封面装订成册。
本要求由研究生院负责解释。
\begin{equation}
	\left(\begin{array}{l}
		x \\
		y \\
	\end{array}\right)=f\left[\left\{\begin{array}{l}
		\xi \\
		\eta \\
	\end{array}\right\}\right]
\end{equation}

\begin{figure}[!ht]
	\centering
	{\includegraphics[scale=0.1]{Figures/logo/ujslogo.pdf}}
	\bicaption{江苏大学}{Jiangsu University}
	\label{Jiangsu University2}
\end{figure}


\begin{table}[h]
	\centering
	\renewcommand{\arraystretch}{1.5} %调整段落行距
	\setlength{\tabcolsep}{15pt} %调整宽度
	\caption{Parameters of flux weakening of the three motors HPM}
	\begin{tabular}{cccc}
		\toprule[1.5pt]
		参数 & 永磁同步电机 I & 永磁同步电机 II & 永磁同步电机 III \\ \midrule[0.75pt]
		初始转矩 \(T_{ini}\) & 27.7Nm & 12.8Nm & 8.6Nm\\
		初始转矩 \(T_{ini}\) & 27.7Nm & 12.8Nm & 8.6Nm \\ \bottomrule[1.5pt]  
	\end{tabular}
\end{table}
}
