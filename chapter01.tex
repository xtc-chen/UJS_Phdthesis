\chapter{绪论}

\section{图表}
\subsection{图}
图应有图号、图题及必要的说明,图中标注一律用英文,图文说明用中文。图应具有“自明性”,即只看图、图题而不阅读正文,就可以理解图意。切忌与表重复。图位于文中表述之后。

图号由“图”和阿拉伯数字组成,阿拉伯数字由前后两部分组成,中间用“.”号分开,前部分数字表示图所在章的序号,后部分数字表示图在该章的序号。例如:“图4.3”,它表示第四章第三个图。

图题采用中英文对照,中文用5号楷体,英文用5号字。图题置于图号之后,图号及图题置于图下方居中位置。

引用图应在图题右上角标出文献来源。

曲线图的纵横坐标必须标注“量、标准规定符号、单位”,此三者只有在不必要标明(如无量纲等)的情况下方可省略。坐标上标注的量的符号和缩略词必须与正文中一致。

照片图要求主题和主要显示部分的轮廓鲜明,便于制版。如用放大缩小的复制品,必须清晰,反差适中。照片上要有表示目的物尺寸的标度。

绘图必须工整、清楚、规范,其中机械零件图按机械制图规格要求,示意图应能清楚反映图示内容。

\begin{figure}[!ht]
	\centering
	{\includegraphics[scale=0.1]{Figures/logo/ujslogo.pdf}}
	\bicaption{江苏大学}{Jiangsu University}
	\label{Jiangsu University}
\end{figure}

\subsection{表}

表应有表号、表题或必要的说明,表也应有“自明性”。表的编排尽可能按内容(或测试项目)由左至右横排、数据依序竖排的格式安排,表的各栏应标明量纲、单位或标准规定的符号,只有在无必要的情况下方可省略。表内不宜用“同上”、“同左”、“‥”或类似的词及符号,表内空白表示“无此项”或“未测”。

表号的编写方法与图号相同,例如“表2.6”,它表示第二章第六个表。

表题置于表号之后,表号及表题置于表上方居中位置。

如某个表需要转页接排,在随后的各页上要重复表号,表号后跟表题(可省略)或跟“(续)”,如表1.2(续)。续表均要重复表的编排。
图、表若需注释,用“(注:……)”(五号楷体)表示,放在相应图或表格下方。

\begin{table*}[!ht]
	\centering
	\bicaption{文献中A处的百分比误差}{Percentage error at A} \label{rotationalframe}
	\scalebox{0.65}{
		\begin{tabular}{lccccccccccc}
			\toprule[1.5pt]
			Load & \multicolumn{2}{c}{$\delta_{p}=\delta_{q}=\delta_{r}=0$} & & \multicolumn{2}{c}{$\delta_{p}=1.0$}  &  & \multicolumn{2}{c}{$\delta_{q}=2.0$}    & &  \multicolumn{2}{c}{$\delta_{r}=3.0$}  \tabularnewline
			\cline{2-3} \cline{5-6} \cline{8-9} \cline{11-12}
			& AAAAAAAA & BBBBBBBB &  & AAAAAAAA & BBBBBBBB &  & AAAAAAAA & BBBBBBBB  &   & AAAAAAAA & BBBBBBBB   \rule{0pt}{15pt} \tabularnewline
			\midrule[0.75pt]
			0   & -2 & −2    &  & -3  & -3  &  & −4 & −4  &   & −5    & −5  \tabularnewline
			\bottomrule[1.5pt]  
	\end{tabular}}
\end{table*}

\begin{table}[h]
	\centering
	\renewcommand{\arraystretch}{1.5} %调整段落行距
	\setlength{\tabcolsep}{15pt} %调整宽度
	\bicaption{HPM}{Parameters of flux weakening of the three motors HPM}
	\begin{tabular}{cccc}
		\toprule[1.5pt]
		参数 & 永磁同步电机 I & 永磁同步电机 II & 永磁同步电机 III \\ \midrule[0.75pt]
		初始转矩 \(T_{ini}\) & 27.7Nm & 12.8Nm & 8.6Nm\\
		初始转矩 \(T_{ini}\) & 27.7Nm & 12.8Nm & 8.6Nm \\ \bottomrule[1.5pt]  
	\end{tabular}
\end{table}


\subsection{公式}
公式书写应在文中另起一行,较长的公式尽可能在等号处回行,或者在“+”、“-”等符号处回行。公式中分数线的横线长短要分清,主要的横线应与等号取平。公式后应注明序号,该序号按章顺序编排,用括弧括起来写在右边行末,其间不加虚线。不得使用“*”等其它符号标记公式序号。   
例如

\begin{equation}\label{transformation}
	\left(\begin{array}{l}
		x \\
		y \\
	\end{array}\right)=f\left[\left\{\begin{array}{l}
		\xi \\
		\eta \\
	\end{array}\right\}\right]
\end{equation}

\begin{align}
	\label{eq:error1}
	\hatvpsi_k(n) &= \vpsi_k(n) - \vw^\star(n) \\
	\label{eq:error2}
	\hatvw_k(n) &= \vw_k(n) - \vw^\star(n)
\end{align}


\begin{equation}\label{test}
	I=\frac VR
\end{equation}

引用此公式时用式\eqref{test}。

\subsection{算法}

算法\cite{cottrell2009}
\begin{enumerate}
	\item[1.] 
	\item[2.] 
	\item[3.] 
\end{enumerate}


\begin{algorithm}[!ht]
	\caption{算法流程图名字} %算法的名字
	\label{algorithm}
	\begin{algorithmic}[!h]
		\State {\textbf{初始化:}} % \State 后写一般语句
		\For{ $n=1,2,\ldots,N$,} % For 语句,需要和EndFor对应
		\State {\textbf{开始步骤:}} 
		\For{每个节点 $k=1,2,\ldots,K$,} % For 语句,需要和EndFor对应
		\EndFor
		\State {\textbf{合并步骤:}} 
		\For{每个节点 $k=1,2,\ldots,K$,} % For 语句,需要和EndFor对应
		\State{ $(n+1) $}
		\EndFor
		\EndFor
	\end{algorithmic}
\end{algorithm}

\begin{remark}
	全部
	都是
	注释
\end{remark}

\begin{prof}
	全部
	都是
	证明
\end{prof}
\begin{theorem}
	全部
	都是
	定理
\end{theorem}
\begin{lemma}
	全部
	都是
	引理
\end{lemma}
\begin{axiom}
	全部
	都是
	公理
\end{axiom}
\begin{proposition}
	全部
	都是
	命题
\end{proposition}
\begin{definition}
	全部
	都是
	定义
\end{definition}
\begin{assumption}
	全部
	都是
	假设
\end{assumption}

\begin{corollary}
	全部
	都是
	推论
\end{corollary}
\begin{exercise}
	全部
	都是
	练习
\end{exercise}

\begin{problem}
	全部
	都是
	问题
\end{problem}
\begin{conjecture}
	全部
	都是
	猜想
\end{conjecture}



